%% This is file `dib-template.tex',
%%
%% Copyright 2020 Elsevier Ltd
%%
%% This file is part of the 'Elsarticle Bundle'.
%% ---------------------------------------------
%%
%% It may be distributed under the conditions of the LaTeX Project Public
%% License, either version 1.2 of this license or (at your option) any
%% later version.  The latest version of this license is in
%%    http://www.latex-project.org/lppl.txt
%% and version 1.2 or later is part of all distributions of LaTeX
%% version 1999/12/01 or later.
%%
%% The list of all files belonging to the 'Elsarticle Bundle' is
%% given in the file `manifest.txt'.
%%
%% Template article for Elsevier's document class `elsarticle'
%% with harvard style bibliographic references
%%
%% $Id: dib-template.tex 185 2020-08-07 09:06:08Z rishi $
%%
%% Use the option review to obtain double line spacing
%\documentclass[times,review,preprint]{elsarticle}

%% Use the options `final' to obtain the final layout
%% Use longtitle option to break abstract to multiple pages if overfull.
%% For Review pdf (With double line spacing)
%\documentclass[times,review]{elsarticle}
%% For abstracts longer than one page.
%\documentclass[times,review,longtitle]{elsarticle}
%% For Review pdf without preprint line
%\documentclass[times,review,nopreprintline]{elsarticle}
%% Final pdf
\documentclass[times,final]{elsarticle}
%%
%\documentclass[times,final,longtitle]{elsarticle}
%%

%%
%% Stylefile to load DIB template
\usepackage{dib}
\usepackage{framed,multirow}

%% The amssymb package provides various useful mathematical symbols
\usepackage{amssymb}
\usepackage{latexsym}

%% For line numbers
%\usepackage[switch]{lineno}

% Following three lines are needed for this document.
% If you are not loading colors or url, then these are
% not required.
\usepackage{url}
\usepackage{xcolor}
\definecolor{newcolor}{rgb}{.8,.349,.1}

%%
\usepackage{longtable}
\usepackage[colorlinks]{hyperref}

\journal{Data in Brief}

\begin{document}

\verso{Given-name Surname \textit{etal}}

\begin{frontmatter}

\dochead{Data Article}
%The article title must include the word 'data' or 'dataset'.
%Please avoid the use of acronyms and abbreviations where possible.
%For co-submission authors, the title should be unique,
%i.e. not the same as your research paper.
%A maximum of 250 characters is allowed.
\title{idealista18: A data package with real estate information in three major Spanish markets from the Idealista database\tnoteref{tnote1}}%
\tnotetext[tnote1]{This is an example for title footnote coding.}
%Tip: here are a few examples of recent suitable article titles - these are short and clear:
%%Adolescent Rat Social Play: Amygdalar Proteomic and Transcriptomic Data
%%Execution Data Logs of a Supercomputer Workload Over its Extended Lifetime
%%Calgary Preschool Magnetic Resonance Imaging (MRI) Dataset]

%%Authors
\author[1]{David \snm{Rey Blanco}}
\author[1]{Pelayo \snm{González Arbues}\fnref{fn1}}
\fntext[fn1]{This is author footnote for second author.}
\author[2]{Fernando \snm{López Hernández}}
\author[3]{Antonio \snm{Páez}\corref{cor1}}
%% Fourth author's email
\ead{paezha@mcmaster.ca}
\cortext[cor1]{Corresponding author:
  Tel.: +1-905-525-9140 ext 26099}

%%Affiliations
\address[1]{Affiliation 1, Address, City and Postal Code, Country}
\address[2]{Affiliation 2, Address, City and Postal Code, Country}
\address[3]{School of Earth, Environment and Society, McMaster University, 1280 Main St W, Hamilton, Ontario L8S 4K1 Canada}

%\received{1 May 2013}
%\finalform{10 May 2013}
%\accepted{13 May 2013}
%\availableonline{15 May 2013}
%\communicated{S. Sarkar}


\begin{abstract}
%%%%
Please Type your abstract here.

\noindent[The Abstract should describe the data collection process, the analysis
performed, the data, and their reuse potential. It should not provide
conclusions or interpretive insights. If your article is being
submitted via another Elsevier journal as a co-submission, please cite
this research article in the abstract.

\noindent\textbf{Tip:} do not use words such as
`study', `results', and `conclusions' because a data article should be
describing your data only.  Min 100 words - Max 500 words.]
%%%%
\end{abstract}

\begin{keyword}
%% Keywords
%[Include 4-8 keywords (or phrases) to facilitate others finding your
%article online.
%\noindent\textbf{Tip:} Try Google Scholar to find which terms are most common in your
%field. In biomedical fields, MeSH terms are a good 'common vocabulary'
%to draw from]
\KWD Property values\sep Spain\sep Spatial analysis\sep Machine learning\sep Hedonic price analysis
\end{keyword}

\end{frontmatter}

%% For linenumbers
%\linenumbers

{\fontsize{7.5pt}{9pt}\selectfont
%%%
\noindent\textbf{Specifications Table}

Every section of this table is mandatory.
Please enter information in the right-hand column and remove all the instructions
\begin{longtable}{|p{33mm}|p{94mm}|}
\hline
\endhead
\hline
\endfoot
Subject                & Geography, Economics\\
\hline
Specific subject area  & Spatial analysis, machine learning, hedonic price analysis\\
\hline
Type of data           & Table\newline\\
%\clearpage
How data were acquired & [State how the data were acquired: E.g. Microscope,
                         SEM, NMR, mass spectrometry, survey* etc.\newline
                         Instruments: E.g. hardware, software, program\newline
                         Make and model and of the instruments used:\newline

                         {\fontsize{7pt}{8pt}\selectfont
                         *\,if you conducted a survey you must submit a copy of the
                         survey(s) used (either provide these as supplementary material
                         file or provide a URL link to the survey
                         in this section of the table).
                         If the survey is not written in English,
                         please provide an English-language translation.}]\\
\hline
Data format            & Spatially masked\\
\hline
Parameters for
data\newline
collection             & [Provide a brief description of which conditions were considered
                         for data collection. Max 400 characters]\\

\hline
Description of
data\newline
collection             & [Provide a brief description of how these data were collected.
                         Max 600 characters]\\
\hline
Data source location   & Institution: Idealista\newline
                         City/Town/Region: Madrid\newline
                         Country: Spain\newline
                         Latitude and longitude (and GPS coordinates, if possible) for collected samples/data:\newline


                         If you are describing secondary data, you are required to provide a list of
                         the primary data sources used in the section.\newline

                         Primary data sources:  ]\\
\hline
\hypertarget{target1}
{Data accessibility}   & Repository name: GitHub\newline
                         Direct URL to data: https://github.com/paezha/idealista18\newline
                         \\
\hline
Related
research\newline
article                & D. Rey Blanco, P. González Arbues, F. López Hernández, A. Páez, Using machine learning to identify spatial market segments: A reproducible study of major Spanish markets, Comput Environ Urban Syst. In Press.\newline
\end{longtable}
}
%%%

\section*{Value of the Data}

[Provide 3-6 bullet points explaining why these data are of value to the scientific community.
Bullet points 1-3 must specifically answer the questions next to the bullet point,
but do not include the question itself in your answer. You may
provide up to three additional bullet points to outline the value of these data.
Please keep points brief, with ideally no more than 400 characters for each point.]

\begin{itemize}
\itemsep=0pt
\parsep=0pt
\item Your first bullet point must explain why these data are useful or important?
\item Your second bullet point must explain who can benefit from these data?
\item Your third point bullet must explain how these data might be used/reused for
further insights and/or development of experiments.
\item In the next three points you may like to explain how these data could
potentially make an impact on society and highlight any other additional value of these data.
\item ....
\end{itemize}

\section*{Data Description}

\noindent [Individually describe each data file (i.e. figure 1, figure 2, table
1, dataset, raw data, supplementary data, etc.) that are included in
this article. Please make sure you refer to every data file and provide
a clear description for each - do not simply list them. No insight,
interpretation, background or conclusions should be included in this
section. Please include legends with any tables, figures or graphs.

\noindent\textbf{Tip:} do not forget to describe any supplementary data files.]

\section*{Experimental Design, Materials and Methods}

\noindent [Offer a complete description of the experimental design and methods
used to acquire these data. Please provide any programs or code files
used for filtering and analyzing these data. It is very important that
this section is as comprehensive as possible. If you are submitting via
another Elsevier journal (a co-submission) you are encouraged to
provide more detail than in your accompanying research article. There
is no character limit for this section; however, no insight,
interpretation, or background should be included in this section.

\noindent\textbf{Tip:} do not describe your data (figures, tables, etc.) in this section,
do this in the Data Description section above.]

\section*{Ethics Statement}
\noindent [Please refer to the journal's
\href{https://www.elsevier.com/journals/data-in-brief/2352-3409/guide-for-authors}{Guide for Authors}
for more information on
the ethical requirements for publication in Data in Brief. In addition
to these requirements:

\noindent\textbf{If the work involved the use of human subjects:}
please include a statement here confirming that informed consent was
obtained for experimentation with human subjects;

\noindent\textbf{If the work involved animal experiments:} please
include a statement confirming that all experiments comply with
the \href{https://www.nc3rs.org.uk/arrive-guidelines}{ARRIVE\ guidelines} and were be carried out in accordance with the
U.K. Animals (Scientific Procedures) Act, 1986 and associated
guidelines, \href{https://ec.europa.eu/environment/chemicals/lab_animals/legislation_en.htm}{EU Directive 2010/63/EU for animal experiments}, or the
National Institutes of Health guide for the care and use of Laboratory
animals (NIH Publications No. 8023, revised 1978)]

\section*{Acknowledgments}
Acknowledgments should be inserted at the end of the paper, before the
references, not as a footnote to the title. Use the unnumbered
Acknowledgements Head style for the Acknowledgments heading.

\section*{Declaration of Competing Interest}

\noindent [All authors are required to report the following information:
\begin{enumerate}
\item[(1)] All third-party financial support for the work this article;

\item[(2)] All financial relationships with any entity that could be
viewed as relevant to data described in this manuscript;

\item[(3)] All sources of revenue with relevance to this work where
payments have been made to authors, or their institutions on their
behalf, within the 36 months prior to submission;

\item[(4)] Any other interactions with the sponsor, outside of the
submitted work;

\item[(5)] Any relevant patents or copyrights (planned, pending or
issued);

\item[(6)] Any other relationships or affiliations that may be
perceived by readers to have influenced, or give the appearance of
potentially influencing, what has been written in this article.
\end{enumerate}
As a general guideline, it is usually better to
disclose a relationship than not. This information will be acknowledged
at publication in the manuscript. If there is no known competing
financial interests or personal relationships that could have appeared
to influence the work reported in this paper, please include this
statement.]

\vskip12pt\noindent
The authors declare that they have no known competing
financial interests or personal relationships which have, or could be
perceived to have, influenced the work reported in this article.

\vskip12pt\noindent
[If there are financial interests/personal relationships which may be
considered as potential competing interests, please declare them here.]

\subsection*{Note}
\label{sec1}
Any instructions relevant to the \verb+elsarticle.cls+ are applicable
here as well. See the online instruction available on:
\makeatletter
\if@twocolumn
\begin{verbatim}
 http://support.river-valley.com/wiki/
 index.php?title=Elsarticle.cls
\end{verbatim}
\else
\begin{verbatim}
 http://support.river-valley.com/wiki/index.php?title=Elsarticle.cls
\end{verbatim}
\fi

\section*{References}

\noindent [References are limited (approx. 15) and excessive self-citation is not
allowed. \textbf{If your data article is co-submitted via another Elsevier
journal, please cite your associated research article here.}

\noindent\textbf{Reference style:}
Text:?Indicate references by number(s) in square brackets in line with
the text. The actual authors can be referred to, but the reference
number(s) must always be given.?

\noindent Example: '..... as demonstrated [3,6]. Barnaby and Jones [8] obtained a different result ....'?

\noindent [Use \verb+\cite+ command to cite a reference list item in text.

\noindent These are examples for reference citations \cite{1}.
\cite{2}.
\cite{4}.]


%% Numbered
%%If
\bibliographystyle{model1-num-names}
\bibliography{refs.bib}



\end{document}

%%
